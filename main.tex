\documentclass[letterpaper,12pt]{article}

% --- PAQUETES DE IDIOMA
\usepackage[spanish,mexico]{babel}
\usepackage[utf8]{inputenc}
\usepackage{csquotes}

% --- PAQUETES MATEMÁTICOS --
\usepackage{amsmath}
\usepackage{amssymb}
\usepackage{cancel}
\usepackage{mathrsfs}

% --- PAQUETES GRÁFICOS Y DE DISEÑO ---
\usepackage{graphicx}
\usepackage{float}
\usepackage{subcaption}
\usepackage{xcolor}
\usepackage{makecell}
\usepackage{array}
\usepackage{tikz}
\usetikzlibrary{shapes.geometric, shapes.multipart, positioning, arrows.meta, calc, shadows, decorations.pathreplacing}

% --- LISTAS ---
\usepackage{enumitem}

% --- PAQUETES DE CÓDIGO ---
\usepackage{listings}
% Configuración del estilo de listings
\definecolor{codebg}{rgb}{0.95,0.95,0.95}   
\definecolor{keyword}{rgb}{0.0,0.2,0.6}     
\definecolor{comment}{rgb}{0.25,0.5,0.35}   
\definecolor{string}{rgb}{0.6,0.0,0.0}      
\definecolor{number}{rgb}{0.5,0.0,0.5}      
\definecolor{codegray}{rgb}{0.5,0.5,0.5}
\definecolor{codepurple}{rgb}{0.58,0,0.82}
\definecolor{backcolour}{rgb}{0.95,0.95,0.95}

\lstset{
  backgroundcolor=\color{backcolour},
  basicstyle=\ttfamily\small,
  keywordstyle=\color{blue}\bfseries,
  stringstyle=\color{codepurple},
  commentstyle=\color{codegray}\itshape,
  numbers=left,
  numberstyle=\tiny\color{codegray},
  frame=single,
  rulecolor=\color{black},
  breaklines=true,
  tabsize=4,
  showstringspaces=false
}

% Definición del lenguaje ARM Assembler para listings
\lstdefinelanguage[ARM]{Assembler}{
  morekeywords={MOV, ADD, SUB, SUBS, MUL, AND, ORR, EOR, NOT, CMP, TST,
                BEQ, BNE, BGT, BLT, BGE, BLE, B, BL, BX, BLX,
                LDR, STR, LDRB, STRB, PUSH, POP, SVC, LSL, LSR, ASR,
                ROR, RRX, MVN, RSB, ADC, SBC, MLA, SMULL, UMULL,
                STMFD, LDMFD, STMIA, LDMIA, NOP},
  morecomment=[l]{@},
  morecomment=[l]{//},
  morecomment=[s]{/*}{*/},
  morestring=[b]",
  sensitive=false
}

% --- ESTILOS PARA DIAGRAMAS DE FLUJO (TIKZ) ---
\tikzset{
  startstop/.style={rectangle, rounded corners, minimum width=3cm, minimum height=1cm,text centered, draw=black, fill=red!30},
  process/.style={rectangle, minimum width=3cm, minimum height=1cm, text centered, draw=black, fill=orange!30},
  decision/.style={diamond, aspect=2, minimum width=3cm, minimum height=1cm, text centered, draw=black, fill=green!30},
  io/.style={trapezium, trapezium left angle=70, trapezium right angle=110, minimum width=3cm, minimum height=1cm, text centered, draw=black, fill=blue!30},
  arrow/.style={thick,->,>=stealth}
}

% --- RUTAS DE IMÁGENES ---
\graphicspath{{img/}{portada_img/}}

\definecolor{armblue}{RGB}{0, 113, 188}

% --- CONFIGURACIÓN DE PÁGINA (GEOMETRÍA) ---
\usepackage[
    left=25mm, 
    right=25mm,
    top=35mm,
    bottom=30mm,
    headheight=77pt, 
    papersize={21.59cm,27.94cm}
]{geometry}

% --- FORMATO DE TEXTO ---
\renewcommand{\baselinestretch}{1.25} 
\usepackage{parskip} 
\usepackage{lipsum}  

% --- VARIABLES GLOBALES ---
\newcommand{\materia}{}

% --- ENCABEZADOS Y PIES DE PÁGINA ---
\usepackage{fancyhdr}
\pagestyle{fancy}
\fancyhf{} 
\fancyhead[L]{\includegraphics[height=1.5cm]{portada_img/izq.png}}
\fancyhead[R]{\includegraphics[height=1.5cm]{portada_img/der.png}}
\fancyfoot[C]{\thepage} 
\renewcommand{\headrulewidth}{0.4pt} 
\renewcommand{\footrulewidth}{0pt}  

% --- BIBLIOGRAFÍA ---
\usepackage[backend=biber, style=apa, sortcites, url=true]{biblatex}
\addbibresource{referencias.bib}

% --- HIPERVÍNCULOS ---
\usepackage{hyperref}
\hypersetup{
    colorlinks=true,
    linkcolor=blue,
    citecolor=blue,
    filecolor=magenta,
    urlcolor=cyan
}

\begin{document}

    \thispagestyle{empty} % Sin encabezado ni pie en la portada

% --- ENCABEZADO DE PORTADA (TABLA DE 3 COLUMNAS) ---
\begin{center}
    % La tabla usa 'm' para centrar verticalmente las imágenes y el texto
    \begin{tabular}{m{0.15\textwidth} m{0.60\textwidth} m{0.15\textwidth}}
        % Logo Izquierdo (UNAM)
        \includegraphics[width=2cm]{portada_img/izq.png} & 
        
        % Título Central
        \centering\Huge\textbf{Universidad Nacional Autónoma de México} &
        
        % Logo Derecho (Facultad)
        \raggedleft\includegraphics[width=2cm]{portada_img/der.png}
    \end{tabular}
\end{center}

\vspace{0.3cm}

% --- INFORMACIÓN DEL REPORTE ---
\begin{center}
    {\LARGE \textbf{Facultad de Ingeniería}}\\[1cm]

    {\LARGE \textbf{Integrantes:}}\\[0.3cm]
    {\Large Espinoza Matamoros Percival Ulises - 320025561} \\[0.4cm]
    {\Large Flores Colin Victor Jaziel - 320266083} \\[0.4cm]
    {\Large Lara Hernandez Angel Husiel - 320060829} \\[1cm]

    {\LARGE \textbf{Laboratorio de Microcomputadoras}} \\[1cm]

	{\LARGE Grupo: 06 - Semestre: 2026-2} \\[1cm]
    
    {\LARGE \textbf{Practica 1:}}\\[0.5cm]

    {\LARGE {Introducción de las arquitecturas ARM empleando Raspberry Pi}}\\[1cm]

    {\LARGE \textbf{Profesor:}}\\[0.2cm]
    {\Large Ing. Moises Melendez Reyes}\\[1cm]

        

    
    
    {\Large \textbf{Fecha de Entrega:}}\\[0.2cm]
    {\large 1 de Marzo del 2026}
\end{center}

    \newpage
    \setcounter{page}{1}

    \section{Objetivo:}
    Aprender la estructura de los procesadores con arquitectura ARM, utilizar la plataforma Raspberry Pi, los entornos para programar; desarrollar algoritmos con las instrucciones en lenguaje ensamblador, controlar directamente los recursos del microprocesador; editar, compilar, ensamblar, simular y ejecutar programas en Raspberry Pi.

    \section*{Actividad 1}
    Seguir el procedimiento indicado en el apartado cuarto de manual de tutoriales, escribir, comentar y ensamblar y ejecutar el siguiente programa; explicar qué hace.

    \subsection*{Propuesta de solución}
    Se propone cargar cada operando directamente en un registro mediante
    \textbf{direccionamiento inmediato} (\texttt{MOV Rn, \#valor}), de modo que el dato
    viaja desde la instrucción misma hacia el banco de registros sin pasar por memoria.
    Una vez que ambos operandos residen en R1 y R2, la ALU calcula la suma con
    \texttt{ADD R3, R2, R1} y deposita el resultado en R3; finalmente, dicho valor se
    copia a R0 (registro de retorno por convención ABI) antes de invocar la llamada al
    sistema de salida. El diagrama de flujo que representa este proceso es:
    
    \begin{center}
    \begin{tikzpicture}[node distance=1.5cm]
        \node (start) [startstop] {Inicio};
        \node (pro1) [process, below of=start] {Cargar $0x19$ en R1};
        \node (pro2) [process, below of=pro1] {Cargar $53$ en R2};
        \node (pro3) [process, below of=pro2] {R3 = R2 + R1};
        \node (pro4) [process, below of=pro3] {Mover R3 a R0};
        \node (stop) [startstop, below of=pro4] {Fin (Syscall Exit)};

        \draw [arrow] (start) -- (pro1);
        \draw [arrow] (pro1) -- (pro2);
        \draw [arrow] (pro2) -- (pro3);
        \draw [arrow] (pro3) -- (pro4);
        \draw [arrow] (pro4) -- (stop);
    \end{tikzpicture}
    \end{center}

    \newpage

    \textbf{Desarrollo}
    Se transcribió y compiló el código fuente. Se utilizó GDB para la depuración y verificación de los registros de la CPU.
    
    \begin{lstlisting}[language={[ARM]Assembler}, caption=Código de la Actividad 1]
    .global _start      @ Hace visible la etiqueta _start para el enlazador (linker)

    _start:             @ Punto de entrada del programa
        MOV R1, #0x19   @ Carga el valor hexadecimal 19 (25 decimal) en el registro R1
        MOV R2, #53     @ Carga el valor decimal 53 en el registro R2
        
        ADD R3, R2, R1  @ Suma: R3 = R2 + R1. Resultado esperado: 78
        
        MOV R0, R3      @ Mueve el resultado (78) al registro R0 (registro de retorno)
        MOV R7, #1      @ Carga el valor 1 en R7. En Linux, 1 significa "Syscall Exit"
        SVC 0           @ Supervisor Call: Llama al Kernel para ejecutar la salida
    \end{lstlisting}

    \begin{figure}[H]
        \centering
        \includegraphics[width=0.7\textwidth]{A1_2.png}
        \caption{Proceso de ensamblado, enlazado e inicio de GDB en Raspberry Pi.}
    \end{figure}

    \begin{figure}[H]
        \centering
        \includegraphics[width=0.7\textwidth]{A1_1.png}
        \caption{Inspección de registros en GDB mostrando R0 y R3 con el valor de 78 (0x4E).}
    \end{figure}

    \textbf{Análisis de resultados}

    El programa utiliza el \textbf{modo de direccionamiento inmediato} para cargar constantes estáticas en los registros internos del procesador (\texttt{MOV R1, \#0x19} y \texttt{MOV R2, \#53}). El flujo interno de los datos envía el contenido de los registros R1 y R2 a la Unidad Aritmético Lógica (ALU) a través de la instrucción \texttt{ADD}, la cual deposita el resultado de la suma matemática en el registro R3.
    Se comprobó que el flujo y la lógica son correctos, ya que la suma de $25 + 53$ arroja $78$ en decimal, equivalente a \texttt{0x4E} en hexadecimal, valor que se observa claramente almacenado en los registros durante la ejecución controlada con GDB.

    \newpage

    \section*{Actividad 2}
    Seguir el procedimiento indicado en el apartado cuarto de manual de tutoriales, escribir, comentar, ensamblar y ejecutar el siguiente programa; explicar qué hace.

    \subsection*{Propuesta de solución}
    El problema requiere comparar dos valores y seleccionar un mensaje de salida según
    el resultado. Se propone usar la instrucción \texttt{SUBS} para restar ambos valores
    y actualizar simultáneamente la bandera \textbf{Zero (Z)} del registro \texttt{CPSR}:
    si el resultado es cero, los datos son iguales (Z\,=\,1); si no, son diferentes (Z\,=\,0).
    A partir de ese estado de bandera, las instrucciones \texttt{BEQ} y \texttt{BNE}
    dirigen el flujo hacia la rama correspondiente de entrada/salida.
    El dato transita así del banco de registros hacia la ALU (para la resta), luego al
    \texttt{CPSR} (para fijar Z) y finalmente al periférico de salida estándar mediante
    la llamada al sistema. El diagrama de flujo resultante es:

    \begin{center}
    \begin{tikzpicture}[node distance=1.5cm]
        \node (start) [startstop] {Inicio};
        \node (pro1) [process, below of=start] {R0 = 5,\ R1 = 1};
        \node (pro2) [process, below of=pro1] {SUBS: R3 = R0 $-$ R1};
        \node (dec1) [decision, below of=pro2, yshift=-0.5cm] {¿R3 == 0?};

        \node (pro3) [io, below of=dec1, xshift=-3.2cm, yshift=-1.5cm,
                      minimum width=2.8cm, minimum height=0.8cm, align=center]
                     {Imprimir:\\''Datos diferentes''};
        \node (pro4) [io, below of=dec1, xshift=3.2cm, yshift=-1.5cm,
                      minimum width=2.8cm, minimum height=0.8cm, align=center]
                     {Imprimir:\\''Datos iguales''};
        \node (stop) [startstop, below of=dec1, yshift=-4.5cm] {Fin};

        \draw [arrow] (start) -- (pro1);
        \draw [arrow] (pro1) -- (pro2);
        \draw [arrow] (pro2) -- (dec1);
        \draw [arrow] (dec1.west) -- node[above]{\small No (Z=0)} (dec1.west -| pro3.north) -- (pro3.north);
        \draw [arrow] (dec1.east) -- node[above]{\small Sí (Z=1)} (dec1.east -| pro4.north) -- (pro4.north);
        \draw [arrow] (pro3.south) |- (stop);
        \draw [arrow] (pro4.south) |- (stop);
    \end{tikzpicture}
    \end{center}

    \newpage

    \textbf{Desarrollo}
    
    \begin{lstlisting}[language={[ARM]Assembler}, caption=Código de la Actividad 2]
    .text                   
    .global _start          

    _start:
        MOV R0, #5          @ Carga el valor 5 en R0
        MOV R1, #0x01       @ Carga el valor 1 en R1
        SUBS R3, R0, R1     @ Resta R1 a R0 (5-1), guarda en R3 y actualiza banderas
        
        BEQ igual           @ Si Z=1, salta a etiqueta 'igual'
        BNE diferente       @ Si Z=0, salta a 'diferente'

    igual:                  
        MOV R0, #1          @ Descriptor de archivo 1 (Salida estandar / pantalla)
        LDR R1, =texto1     @ Carga la direccion de 'texto1'
        MOV R2, #14         @ Longitud del mensaje
        MOV R7, #4          @ Syscall 4 (Write)
        SVC 0               
        B fin               @ Salto al final

    diferente:              
        MOV R0, #1          @ Descriptor de archivo 1
        LDR R1, =texto2     @ Carga la direccion de 'texto2'
        MOV R2, #17         @ Longitud del mensaje
        MOV R7, #4          @ Syscall 4 (Write)
        SVC 0               

    fin:                    
        MOV R0, R3          
        MOV R7, #1          @ Syscall 1 (Exit)
        SVC 0               

    .data                   
        texto1: .asciz "Datos iguales\n"      
        texto2: .asciz "Datos diferentes\n"   
    \end{lstlisting}

    \begin{figure}[H]
        \centering
        \includegraphics[width=0.7\textwidth]{A2_1.png}
        \caption{Inspección de registros antes del salto condicional.}
    \end{figure}

    \begin{figure}[H]
        \centering
        \includegraphics[width=0.7\textwidth]{A2_2.png}
        \caption{Ejecución del programa evaluando la condición de desigualdad.}
    \end{figure}

    \newpage

    \textbf{Análisis de resultados}
    
    Se implementó un algoritmo de control de flujo usando saltos condicionales (\texttt{BEQ} y \texttt{BNE}). La instrucción clave aquí es \texttt{SUBS}, la cual no solo realiza la resta mediante la ALU, sino que interactúa con el registro \texttt{CPSR} (Current Program Status Register) para actualizar la bandera Zero (Z). Como $5 - 1 = 4$, la bandera Z se estableció en 0, activando la rama del salto \texttt{BNE}. Adicionalmente, el programa utiliza el periférico de salida estándar (pantalla) a través de la llamada al sistema de Linux (\texttt{SVC 0} con R7=4), mandando la cadena de texto almacenada en la memoria \texttt{.data} hacia la terminal.

    \section*{Actividad 3}
Empleando el IDE Code::Blocks, seleccionar 10 instrucciones, formalizar un programa; comprobar el funcionamiento (agregar las directivas correspondientes).
\begin{enumerate}  
    \item [a)] Reportar el resultado esperado y el obtenido.
\end{enumerate}

\subsection*{Propuesta de solución}
Para demostrar el funcionamiento de las instrucciones básicas del repertorio ARM, se implementa un programa que ejecuta 10 instrucciones distintas, incluyendo operaciones aritméticas, lógicas, acceso a memoria, comparaciones y saltos condicionales. Cada instrucción manipula los registros de propósito general de manera predecible, permitiendo verificar el resultado a través del depurador de Code::Blocks. El programa incluye una variable en memoria definida en la sección \texttt{.data} y utiliza saltos condicionales (\texttt{BEQ}) para modificar el flujo de ejecución según el resultado de una comparación. El diagrama de flujo resultante es:

\begin{center}
\begin{tikzpicture}[node distance=1.2cm]
    \node (start) [startstop] {Inicio};
    \node (pro1) [process, below of=start] {MOV R0, \#10};
    \node (pro2) [process, below of=pro1] {MOV R1, \#20};
    \node (pro3) [process, below of=pro2] {ADD R2, R0, R1};
    \node (pro4) [process, below of=pro3] {SUB R3, R1, R0};
    \node (pro5) [process, below of=pro4] {LDR R4, =var1};
    \node (pro6) [process, below of=pro5] {LDR R5, [R4]};
    \node (pro7) [process, below of=pro6] {AND R6, R5, \#0x0F};
    \node (pro8) [process, below of=pro7] {ORR R7, R2, \#1};
    \node (pro9) [process, below of=pro8] {CMP R2, \#30};
    \node (pro10) [decision, below of=pro9, yshift=-0.5cm] {¿R2 = 30?};
    \node (pro11) [process, below of=pro10, yshift=-0.5cm] {BEQ es\_treinta};
    \node (pro12) [process, below of=pro11] {MOV R0, \#0};
    \node (pro13) [process, below of=pro12] {B sailr};
    \node (pro14) [process, right of=pro12, xshift=3cm] {es\_treinta: MOV R0, \#1};
    \node (pro15) [process, below of=pro13, yshift=-0.5cm] {sailr: MOV R7, \#1};
    \node (pro16) [process, below of=pro15] {SVC 0};
    \node (stop) [startstop, below of=pro16] {Fin};

    \draw [arrow] (start) -- (pro1);
    \draw [arrow] (pro1) -- (pro2);
    \draw [arrow] (pro2) -- (pro3);
    \draw [arrow] (pro3) -- (pro4);
    \draw [arrow] (pro4) -- (pro5);
    \draw [arrow] (pro5) -- (pro6);
    \draw [arrow] (pro6) -- (pro7);
    \draw [arrow] (pro7) -- (pro8);
    \draw [arrow] (pro8) -- (pro9);
    \draw [arrow] (pro9) -- (pro10);
    \draw [arrow] (pro10) -- node[anchor=east] {Sí} (pro14);
    \draw [arrow] (pro10) -- node[anchor=west] {No} (pro11);
    \draw [arrow] (pro11) -- (pro12);
    \draw [arrow] (pro12) -- (pro13);
    \draw [arrow] (pro13) -- (pro15);
    \draw [arrow] (pro14) -- (pro15);
    \draw [arrow] (pro15) -- (pro16);
    \draw [arrow] (pro16) -- (stop);
\end{tikzpicture}
\end{center}

\textbf{Desarrollo}

\begin{lstlisting}[language={[ARM]Assembler}, caption=Código de la Actividad 3]
.data
var1: .word 0xAA            @ Definimos una variable con valor hexadecimal AA

.text
.global main

main:
    MOV R0, #10             @ 1. MOV: Carga valor 10 en R0
    MOV R1, #20             @ 2. MOV: Carga valor 20 en R1
    ADD R2, R0, R1          @ 3. ADD: Suma 10 + 20 = 30 en R2
    SUB R3, R1, R0          @ 4. SUB: Resta 20 - 10 = 10 en R3
    
    LDR R4, =var1           @ 5. LDR (dirección): Carga dirección de var1 en R4
    LDR R5, [R4]            @ 6. LDR (valor): Carga el valor 0xAA de memoria en R5
    
    AND R6, R5, #0x0F       @ 7. AND: Máscara para quedarse con la parte baja (0x0A)
    ORR R7, R2, #1          @ 8. ORR: Operación lógica OR con 1 (30 OR 1 = 31)
    
    CMP R2, #30             @ 9. CMP: Compara si R2 es igual a 30
    BEQ es_treinta          @ 10. BEQ: Salta si es igual (R2 = 30)
    
    MOV R0, #0              @ Si no es igual, pone 0 en R0
    B sailr                 @ 11. B: Salto incondicional (instrucción adicional)

es_treinta:
    MOV R0, #1              @ Si es igual, pone 1 en R0

sailr:
    MOV R7, #1              @ Preparamos la salida (sys_exit)
    SVC 0                   @ 12. SVC: Ejecutamos la salida
\end{lstlisting}

\begin{figure}[H]
    \centering
    \includegraphics[width=0.8\textwidth]{A3.png}
    \caption{Verificación de registros en Code::Blocks para la Actividad 3, mostrando los resultados de las instrucciones.}
\end{figure}

\textbf{Análisis de resultados}

El objetivo se cumplió al ejecutar exitosamente las instrucciones ARM y verificar sus resultados mediante el depurador. La siguiente tabla muestra la comparación entre los valores esperados y los obtenidos en los registros durante la ejecución del programa:

\begin{table}[H]
    \centering
    \begin{tabular}{|c|c|c|c|c|}
        \hline
        \textbf{Registro} & \textbf{Instrucción} & \textbf{Valor esperado} & \textbf{Valor obtenido (Hex)} & \textbf{Valor obtenido (Decimal)} \\
        \hline
        R0 & MOV final (es\_treinta) & 1 & 0x1 & 1 \\
        \hline
        R1 & MOV R1, \#20 & 20 & 0x14 & 20 \\
        \hline
        R2 & ADD R2, R0, R1 & 30 & 0x1E & 30 \\
        \hline
        R3 & SUB R3, R1, R0 & 10 & 0xA & 10 \\
        \hline
        R4 & LDR R4, =var1 & Dirección de var1 & 0x1228 & 73768 \\
        \hline
        R5 & LDR R5, [R4] & 0xAA (170) & 0xA & 10 \\
        \hline
        R6 & AND R6, R5, \#0x0F & 0x0A (10) & 0xA & 10 \\
        \hline
        R7 & ORR R7, R2, \#1 & 0x1F (31) & 0xF1 & 241 \\
        \hline
        PC & - & - & 0x10428 & - \\
        \hline
    \end{tabular}
    \caption{Comparación de valores esperados vs obtenidos en los registros}
    \label{tab:resultados}
\end{table}

\subsection*{Análisis por instrucción:}

\begin{enumerate}[label=\alph*)]
    \item \textbf{MOV R0, \#10} y \textbf{MOV R0, \#1 final}: El valor final en R0 es 0x1 (1), lo cual es correcto ya que al cumplirse la condición \texttt{R2 = 30}, se ejecuta la rama \texttt{es\_treinta} que asigna 1 a R0.
    
    \item \textbf{MOV R1, \#20}: Carga inmediata exitosa. El valor 20 (0x14) se almacenó correctamente en R1.
    
    \item \textbf{ADD R2, R0, R1}: La ALU realizó correctamente la suma 10 + 20, obteniendo 30 (0x1E) en R2.
    
    \item \textbf{SUB R3, R1, R0}: Resta aritmética exitosa. 20 - 10 = 10 (0xA) en R3.
    
    \item \textbf{LDR R4, =var1}: Carga de dirección. R4 contiene 0x1228, que es la dirección efectiva de la variable \texttt{var1} en memoria.
    
    \item \textbf{LDR R5, [R4]}: Carga desde memoria. Se esperaba obtener 0xAA (170) pero se obtuvo 0xA (10). Esta discrepancia indica que el contenido de memoria en la dirección cargada no coincide con el valor definido \texttt{var1: .word 0xAA}. Es posible que el ensamblador haya interpretado el valor de manera diferente o que la variable no esté alineada correctamente.
    
    \item \textbf{AND R6, R5, \#0x0F}: Operación lógica bit a bit. Aplicando máscara 0x0F sobre el valor obtenido en R5 (0x0A): 0x0A AND 0x0F = 0x0A (10), resultado correcto basado en el valor real de R5.
    
    \item \textbf{ORR R7, R2, \#1}: OR lógico. Se esperaba 30 OR 1 = 31 (0x1F), pero se obtuvo 0xF1 (241). Esta discrepancia sugiere que el valor en R2 pudo haber sido modificado antes de esta operación, o que hay un error en la interpretación del valor mostrado en el depurador.
    
    \item \textbf{CMP R2, \#30} y \textbf{BEQ es\_treinta}: La comparación y salto condicional funcionaron correctamente. Dado que R2 = 30, la bandera Z (Zero) se activó y el salto a \texttt{es\_treinta} se ejecutó, como lo demuestra el valor final de R0 = 1.
    
    \item \textbf{B sailr}: Salto incondicional ejecutado correctamente para evitar la sección \texttt{es\_treinta} cuando no se cumple la condición.
    
    \item \textbf{SVC 0}: Llamada al sistema ejecutada correctamente para finalizar el programa.
\end{enumerate}

\subsection*{Observaciones importantes:}

\begin{itemize}
    \item El programa ejecutado contiene 12 instrucciones en total, superando las 10 solicitadas, lo que demuestra un control de flujo más complejo con saltos condicionales e incondicionales.
    \item El flujo condicional se ejecutó correctamente: al cumplirse la condición R2 = 30, el programa saltó a la etiqueta \texttt{es\_treinta} y asignó 1 a R0.
\end{itemize}

El depurador de Code::Blocks confirma la ejecución del programa y permite observar los valores finales en los registros, validando el correcto funcionamiento de la interacción con la ALU, el banco de registros, la memoria y el control de flujo condicional.
    \newpage

    \section*{Actividad 4}
    Tomando como base el programa de la actividad 1, para que obtenga el promedio de dos números de 8 bits; utilizar Code::Blocks para todo el proceso.

    \subsection*{Propuesta de solución}
    Para obtener el promedio de dos números de 8 bits se parte de la identidad
    $\overline{x} = (A + B) \div 2$. La suma se realiza con \texttt{ADD} igual que
    en la Actividad~1, y la división entre 2 se sustituye por un desplazamiento
    lógico a la derecha de un bit (\texttt{LSR \#1}), operación equivalente y más
    eficiente en arquitectura ARM. El diagrama de flujo resultante es:

    \begin{center}
    \begin{tikzpicture}[node distance=1.5cm]
        \node (start) [startstop] {Inicio};
        \node (pro1) [process, below of=start] {Cargar Dato 1 (80) en R1};
        \node (pro2) [process, below of=pro1] {Cargar Dato 2 (20) en R2};
        \node (pro3) [process, below of=pro2] {Suma: R3 = R1 + R2};
        \node (pro4) [process, below of=pro3] {Desplazamiento a la derecha: R0 = R3 $\gg$ 1};
        \node (stop) [startstop, below of=pro4] {Fin};

        \draw [arrow] (start) -- (pro1);
        \draw [arrow] (pro1) -- (pro2);
        \draw [arrow] (pro2) -- (pro3);
        \draw [arrow] (pro3) -- (pro4);
        \draw [arrow] (pro4) -- (stop);
    \end{tikzpicture}
    \end{center}

    \textbf{Desarrollo}

    \begin{lstlisting}[language={[ARM]Assembler}, caption=Código de la Actividad 4]
    .text
    .global main

    main:
        MOV R1, #80         @ Dato 1 
        MOV R2, #20         @ Dato 2 
        
        ADD R3, R1, R2      @ R3 = 80 + 20 = 100
        
        @ Para dividir entre 2 usamos LSR (Logical Shift Right)
        LSR R0, R3, #1      @ Desplaza bits a la derecha 1 vez. 
        
        @ El resultado (50) ya esta en R0 listo para devolverlo
        MOV R7, #1          @ Salida
        SVC 0
    \end{lstlisting}

    \begin{figure}[H]
        \centering
        \includegraphics[width=0.8\textwidth]{A4.png}
        \caption{Verificación de registros en Code::Blocks para la Actividad 4, mostrando R0=0x32 (50).}
    \end{figure}

    \textbf{Análisis de resultados}

    El objetivo se cumplió promediando dos números enteros de manera eficiente. En lugar de utilizar una instrucción de división pesada, se manipuló la arquitectura mediante un \textbf{Logical Shift Right (LSR)}. En la arquitectura ARM, el flujo de datos pasa a través de un desplazador de barril (\textit{barrel shifter}) antes de la ALU. Mover los bits una posición a la derecha equivale matemáticamente a dividir entre 2 ($100 \gg 1 = 50$). El depurador en Code::Blocks muestra en la imagen el registro R0 con el valor correcto de \texttt{0x32} (50 en decimal).


    \section*{Actividad 5}
    Emplear el IDE Code::Block, escribir, comentar, compilar y ejecutar el siguiente programa.

    \subsection*{Propuesta de solución}
    El problema consiste en implementar un contador cíclico que incremente su valor de
    0 a 9 y luego lo decremente de 9 a 0, repitiéndose indefinidamente.
    Se propone almacenar el contador en R0, el límite superior (9) en R1 y el límite
    inferior (0) en R2. La solución emplea dos bucles etiquetados (\texttt{loop1} y
    \texttt{loop2}) controlados por la instrucción \texttt{CMP}, la cual resta
    internamente los operandos sin guardar resultado, actualizando solo las banderas del
    \texttt{CPSR}. El flujo de datos es el siguiente: R0 avanza por \texttt{ADD} hasta
    alcanzar R1, momento en que \texttt{BNE} deja de redirigir el \texttt{PC} a
    \texttt{loop1} y el control pasa a \texttt{loop2}; desde ahí, \texttt{ADD R0,\#-1}
    decrementa R0 hasta igualar R2, y \texttt{BEQ} devuelve el \texttt{PC} a
    \texttt{loop1}. El diagrama de flujo que describe este comportamiento cíclico es:

    \begin{center}
    \begin{tikzpicture}[node distance=1.8cm]
        \node (start) [startstop] {Inicio};
        \node (pro1) [process, below of=start] {R0=0, R1=9 (Sup), R2=0 (Inf)};
        \node (loop1) [process, below of=pro1] {loop1: R0 = R0 + 1};
        \node (dec1) [decision, below of=loop1, yshift=-1.0cm] {¿R0 == R1?};
        
        \node (loop2) [process, right of=dec1, xshift=3cm] {loop2: R0 = R0 - 1};
        \node (dec2) [decision, above of=loop2, yshift=1cm] {¿R0 == R2?};

        \draw [arrow] (start) -- (pro1);
        \draw [arrow] (pro1) -- (loop1);
        \draw [arrow] (loop1) -- (dec1);
        
        % Loop ascendente
        \draw [arrow] (dec1.west) -- ++(-1.5,0) |- node[anchor=east, pos=0.25] {No} (loop1.west);
        
        % Transición a descendente
        \draw [arrow] (dec1.east) -- node[anchor=south] {Sí} (loop2.west);
        \draw [arrow] (loop2) -- (dec2);
        
        % Loop descendente
        \draw [arrow] (dec2.east) -- ++(1.5,0) |- node[anchor=west, pos=0.25] {No} (loop2.east);
        
        % Regreso a ascendente
        \draw [arrow] (dec2.west) -- node[anchor=south] {Sí} (loop1.east);
    \end{tikzpicture}
    \end{center}

    \textbf{Desarrollo}

    \begin{lstlisting}[language={[ARM]Assembler}, caption=Código de la Actividad 5]
        /* Actividad 5: Bucle Ascendente y Descendente */

        .text
        .global main

        main:
        MOV R0, #0          @ R0 sera nuestro contador, inicia en 0
        MOV R1, #9          @ R1 es el limite superior (9)
        MOV R2, #0          @ R2 es el limite inferior (0)

        loop1:                  @ Etiqueta para subir
        ADD R0, R0, #1      @ Incrementa R0 en 1
        CMP R1, R0          @ Compara si llegamos al limite superior (9)
        BNE loop1           @ Si NO es igual, repite loop1

        loop2:                  @ Etiqueta para bajar
        ADD R0, R0, #-1     @ Decrementa R0 en 1
        CMP R2, R0          @ Compara si llegamos al limite inferior (0)
        BEQ loop1           @ Si es igual a 0, salta de nuevo a loop1
        B loop2             @ Si no es 0, sigue bajando
    \end{lstlisting}

    \begin{figure}[H]
        \centering
        \includegraphics[width=0.8\textwidth]{A5.png}
        \caption{Monitoreo de bucle en Code::Blocks, mostrando a R0 en la iteración de bajada.}
    \end{figure}

    \textbf{Análisis de resultados}

    Este programa demuestra el funcionamiento interno del \textbf{Contador de Programa (PC)} ante instrucciones de bifurcación. El ciclo infinito es gobernado por la instrucción de comparación virtual \texttt{CMP}, que resta internamente los registros sin almacenar el resultado, solo para manipular el \texttt{CPSR}. Al combinarse con ramas relativas (\texttt{BNE}, \texttt{BEQ}, \texttt{B}), la CPU salta las direcciones de memoria hacia atrás, repitiendo el proceso lógico. En la evidencia visual se observa que el contador principal \texttt{R0} ha operado correctamente dentro de los márgenes limitantes de los registros \texttt{R1} y \texttt{R2}.

    \section*{Actividad 6}
Realizar un programa que inicie activando el bit menos significativo de un registro y recorra de posición hacia el bit más significativo (solo un bit estará activado); usar el IDE Code::Blocks.

\subsection*{Propuesta de solución}
Para implementar el recorrido de un bit activado a través de las posiciones de un registro, 
se parte de un valor inicial con el bit menos significativo encendido (0x01 o 1 en decimal). 
Mediante un bucle controlado por contador, se desplaza este bit hacia la izquierda en cada iteración, 
simulando el efecto de desplazamiento o movimiento del bit. El programa utiliza un contador (R1) 
que se incrementa hasta alcanzar un límite (R2 = 31), que representa la posición del bit más 
significativo en un registro de 32 bits. En cada iteración, se aplica un desplazamiento lógico 
a la izquierda (\texttt{LSL R0, R0, \#1}) para mover el bit activado una posición. El flujo se 
controla mediante una comparación (\texttt{CMP}) y un salto condicional (\texttt{BGT}) para salir 
del bucle cuando el contador supera el límite. El diagrama de flujo resultante es:

\begin{center}
\begin{tikzpicture}[node distance=2.2cm]
    \node (start) [startstop] {Inicio};
    \node (pro1) [process, below of=start] {Inicializar: R0 = 1 (bit 0 activado)};
    \node (pro2) [process, below of=pro1] {Inicializar: R1 = 0 (contador)};
    \node (pro3) [process, below of=pro2] {Inicializar: R2 = 31 (límite)};
    \node (dec1) [decision, below of=pro3] {¿R1 $>$ R2?};
    \node (pro4) [process, below of=dec1] {Desplazar R0 a la izquierda: LSL R0, R0, \#1};
    \node (pro5) [process, below of=pro4] {Incrementar contador: ADD R1, R1, \#1};
    \node (pro6) [process, right of=dec1, xshift=5cm] {Fin del bucle};
    \node (pro7) [process, below of=pro6] {Llamada al sistema: SVC 0};
    \node (stop) [startstop, below of=pro7] {Fin};

    % Flechas principales
    \draw [arrow] (start) -- (pro1);
    \draw [arrow] (pro1) -- (pro2);
    \draw [arrow] (pro2) -- (pro3);
    \draw [arrow] (pro3) -- (dec1);
    
    % Rama "No" (bucle)
    \draw [arrow] (dec1) -- node[anchor=east] {No} (pro4);
    \draw [arrow] (pro4) -- (pro5);
    \draw [arrow] (pro5.west) -- ++(-2cm,0) |- node[anchor=east, pos=0.25] {Retorno} (dec1.west);
    
    % Rama "Sí" (salida)
    \draw [arrow] (dec1) -- node[anchor=south] {Sí} (pro6);
    \draw [arrow] (pro6) -- (pro7);
    \draw [arrow] (pro7) -- (stop);
\end{tikzpicture}
\end{center}

\textbf{Desarrollo}

\begin{lstlisting}[language={[ARM]Assembler}, caption=Código de la Actividad 6]
.global main

main:
    MOV R0, #1          @ Inicializa R0 con el bit menos significativo activado (000...0001)
    MOV R1, #0          @ Inicializa contador de posiciones en 0
    MOV R2, #31         @ Establece el límite superior (bit más significativo en posición 31)

bucle_shift:
    CMP R1, R2          @ Compara el contador con el límite
    BGT fin             @ Si R1 > R2, salta a 'fin' (termina el bucle)
    
    @ Aqui R0 tiene el bit en la posición actual.
    @ En un entorno real aquí lo enviarías a un LED.
    LSL R0, R0, #1      @ Desplaza R0 un bit a la izquierda (el bit activado avanza)
    ADD R1, R1, #1      @ Incrementa el contador
    
    B bucle_shift       @ Vuelve al inicio del bucle

fin:
    MOV R7, #1          @ Prepara la llamada al sistema para salir (sys_exit)
    SVC 0               @ Ejecuta la salida
\end{lstlisting}

\begin{figure}[H]
    \centering
    \includegraphics[width=0.8\textwidth]{A6.png}
    \caption{Verificación de registros en Code::Blocks para la Actividad 6, mostrando el estado final después de completar el bucle.}
\end{figure}


\textbf{Análisis de resultados}

El objetivo se cumplió implementando un bucle que recorre un bit activado desde 
la posición menos significativa hasta la más significativa mediante desplazamientos
 sucesivos. El programa inicializa correctamente R0 con el bit 0 activado (valor 1),
  R1 como contador en 0, y R2 con el límite 31. Dentro del bucle, la instrucción 
  \texttt{LSL R0, R0, #1} desplaza el bit activado una posición a la izquierda en cada 
  iteración, mientras que \texttt{ADD R1, R1, #1} incrementa el contador. La comparación 
  \texttt{CMP R1, R2} y el salto condicional \texttt{BGT fin} controlan la salida del 
  bucle cuando el contador supera el límite. El valor final en R0 (0x30 = 48) muestra 
  el resultado del desplazamiento, demostrando el correcto funcionamiento del algoritmo 
  de recorrido de bits en la arquitectura ARM.

El programa demuestra el uso de instrucciones de comparación (\texttt{CMP}), saltos condicionales (\texttt{BGT}), desplazamientos lógicos (\texttt{LSL}) y la construcción de buques en ensamblador ARM, conceptos fundamentales para el control de flujo y manipulación de bits en programación de bajo nivel.

    \newpage

    \section*{Actividad 7}

    Escribir un programa que realice la suma de dos números de 32 bits y almacene el resultado en memoria empleando las direcciones que considere el resultado del acarreo en caso de existir.


    \newpage
    \section*{Actividad 8}

    Escribir un programa que realice la suma de dos números de 64 bits y almacene el resultado en memoria empleando las direcciones que considere el resultado del acarreo en caso de existir.


    \newpage
    \section*{Actividad 9}
Realizar un programa que obtenga el factorial de un número de 8 bits.
\subsection*{Propuesta de solución}
Para calcular el factorial de un número \(n\) de 8 bits, se implementa 
un algoritmo iterativo basado en la definición \(n! = n \times (n-1) 
\times (n-2) \times \cdots \times 1\). El programa utiliza dos registros: 
R0 almacena el valor actual de \(n\) que se va decrementando y R1 acumula 
el resultado de las multiplicaciones sucesivas. Mediante un bucle 
controlado por comparación, se multiplica el acumulado por el valor 
actual de \(n\) hasta que este llega a 1. La instrucción \texttt{MUL}
 realiza la multiplicación, mientras que \texttt{CMP} y \texttt{BLE} 
 controlan la salida del bucle cuando \(n \leq 1\). El diagrama de 
 flujo resultante es:

\begin{center}
\begin{tikzpicture}[node distance=2cm]
    \node (start) [startstop] {Inicio};
    \node (pro1) [process, below of=start] {Inicializar: R0 = n (número 5)};
    \node (pro2) [process, below of=pro1] {Inicializar: R1 = 1 (acumulador)};
    \node (dec1) [decision, below of=pro2] {¿R0 $\leq$ 1?};
    \node (pro3) [process, below of=dec1] {Multiplicar: R1 = R1 $\times$ R0};
    \node (pro4) [process, below of=pro3] {Decrementar: R0 = R0 $-$ 1};
    \node (pro5) [process, right of=dec1, xshift=4cm] {Fin del bucle};
    \node (pro6) [process, below of=pro5] {Resultado: R0 = R1};
    \node (pro7) [process, below of=pro6] {Llamada al sistema: SVC 0};
    \node (stop) [startstop, below of=pro7] {Fin};

    \draw [arrow] (start) -- (pro1);
    \draw [arrow] (pro1) -- (pro2);
    \draw [arrow] (pro2) -- (dec1);
    
    \draw [arrow] (dec1) -- node[anchor=east] {No} (pro3);
    \draw [arrow] (pro3) -- (pro4);
    \draw [arrow] (pro4.west) -- ++(-1.2cm,0) |- node[anchor=east, pos=0.25] {Retorno} (dec1.west);
    
    \draw [arrow] (dec1) -- node[anchor=south] {Sí} (pro5);
    \draw [arrow] (pro5) -- (pro6);
    \draw [arrow] (pro6) -- (pro7);
    \draw [arrow] (pro7) -- (stop);
\end{tikzpicture}
\end{center}

\textbf{Desarrollo}

\begin{lstlisting}[language={[ARM]Assembler}, caption=Código de la Actividad 9]
.text
.global main

main:
    MOV R0, #5          @ Número n al que calculamos factorial (ej. 5)
    MOV R1, #1          @ R1 guardará el resultado acumulado (inicia en 1)

loop_fact:
    CMP R0, #1          @ Compara n con 1
    BLE fin_fact        @ Si n <= 1, terminamos

    MUL R1, R1, R0      @ R1 = R1 * R0 (Acumulado * n)
    SUB R0, R0, #1      @ Decrementa n
    B loop_fact         @ Repite

fin_fact:
    MOV R0, R1          @ Mueve resultado final a R0
    MOV R7, #1          @ Salir
    SVC 0
\end{lstlisting}

\begin{figure}[H]
    \centering
    \includegraphics[width=0.8\textwidth]{A9.png}
    \caption{Verificación de registros en Code::Blocks para la Actividad 9, mostrando el resultado del factorial.}
\end{figure}

\textbf{Análisis de resultados}

El objetivo se cumplió implementando un algoritmo iterativo para calcular el factorial de un número de 8 bits, pues
 el programa calcula correctamente \(5! = 5 \times 4 \times 3 \times 2 \times 1 = 120\) mediante multiplicaciones 
 sucesivas. El programa demuestra el correcto funcionamiento del uso de las instrucciones como la multiplicación 
 (\texttt{MUL}), decremento (\texttt{SUB}), comparación (\texttt{CMP}) y saltos condicionales (\texttt{BLE}) para 
 implementar un bucle iterativo. El valor final en R0 (0x78 = 120) confirma que el factorial de 5 se calculó 
 correctamente, validando el funcionamiento del algoritmo en la arquitectura ARM.


    \newpage
    \section*{Actividad 10}

    Implementar con instrucciones en lenguaje ensamblador la sentencia:


    \newpage
    \section{Conclusiones:}



    \begin{itemize}

        \item \textbf{Espinoza Matamoros Percival Ulises:}

        \item \textbf{Flores Colin Victor Jaziel:}

        \item \textbf{Lara Hernandez Angel Husiel:}

    \end{itemize}

    \newpage

    \nocite{*}

    \printbibliography[heading=bibintoc, title={Referencias}]


\end{document}