\documentclass[letterpaper,12pt]{article}

% --- PAQUETES DE IDIOMA
\usepackage[spanish,mexico]{babel}
\usepackage[utf8]{inputenc}
\usepackage{csquotes}

% --- PAQUETES MATEMÁTICOS --
\usepackage{amsmath}
\usepackage{amssymb}
\usepackage{cancel}
\usepackage{mathrsfs}

% --- PAQUETES GRÁFICOS Y DE DISEÑO ---
\usepackage{graphicx}
\usepackage{float}
\usepackage{subcaption}
\usepackage{xcolor}
\usepackage{makecell}
\usepackage{array}
\usepackage{tikz}
\usetikzlibrary{shapes.geometric, shapes.multipart, positioning, arrows.meta, calc, shadows, decorations.pathreplacing}

% --- LISTAS ---
\usepackage{enumitem}

% --- PAQUETES DE CÓDIGO ---
\usepackage{listings}
% Configuración del estilo de listings
\definecolor{codebg}{rgb}{0.95,0.95,0.95}   
\definecolor{keyword}{rgb}{0.0,0.2,0.6}     
\definecolor{comment}{rgb}{0.25,0.5,0.35}   
\definecolor{string}{rgb}{0.6,0.0,0.0}      
\definecolor{number}{rgb}{0.5,0.0,0.5}      
\definecolor{codegray}{rgb}{0.5,0.5,0.5}
\definecolor{codepurple}{rgb}{0.58,0,0.82}
\definecolor{backcolour}{rgb}{0.95,0.95,0.95}

\lstset{
  backgroundcolor=\color{backcolour},
  basicstyle=\ttfamily\small,
  keywordstyle=\color{blue}\bfseries,
  stringstyle=\color{codepurple},
  commentstyle=\color{codegray}\itshape,
  numbers=left,
  numberstyle=\tiny\color{codegray},
  frame=single,
  rulecolor=\color{black},
  breaklines=true,
  tabsize=4,
  showstringspaces=false
}

% Definición del lenguaje ARM Assembler para listings
\lstdefinelanguage[ARM]{Assembler}{
  morekeywords={MOV, ADD, SUB, SUBS, MUL, AND, ORR, EOR, NOT, CMP, TST,
                BEQ, BNE, BGT, BLT, BGE, BLE, B, BL, BX, BLX,
                LDR, STR, LDRB, STRB, PUSH, POP, SVC, LSL, LSR, ASR,
                ROR, RRX, MVN, RSB, ADC, SBC, MLA, SMULL, UMULL,
                STMFD, LDMFD, STMIA, LDMIA, NOP},
  morecomment=[l]{@},
  morecomment=[l]{//},
  morecomment=[s]{/*}{*/},
  morestring=[b]",
  sensitive=false
}

% --- ESTILOS PARA DIAGRAMAS DE FLUJO (TIKZ) ---
\tikzset{
  startstop/.style={rectangle, rounded corners, minimum width=3cm, minimum height=1cm,text centered, draw=black, fill=red!30},
  process/.style={rectangle, minimum width=3cm, minimum height=1cm, text centered, draw=black, fill=orange!30},
  decision/.style={diamond, aspect=2, minimum width=3cm, minimum height=1cm, text centered, draw=black, fill=green!30},
  io/.style={trapezium, trapezium left angle=70, trapezium right angle=110, minimum width=3cm, minimum height=1cm, text centered, draw=black, fill=blue!30},
  arrow/.style={thick,->,>=stealth}
}

% --- RUTAS DE IMÁGENES ---
\graphicspath{{img/}{portada_img/}}

\definecolor{armblue}{RGB}{0, 113, 188}

% --- CONFIGURACIÓN DE PÁGINA (GEOMETRÍA) ---
\usepackage[
    left=25mm, 
    right=25mm,
    top=35mm,
    bottom=30mm,
    headheight=77pt, 
    papersize={21.59cm,27.94cm}
]{geometry}

% --- FORMATO DE TEXTO ---
\renewcommand{\baselinestretch}{1.25} 
\usepackage{parskip} 
\usepackage{lipsum}  

% --- VARIABLES GLOBALES ---
\newcommand{\materia}{}

% --- ENCABEZADOS Y PIES DE PÁGINA ---
\usepackage{fancyhdr}
\pagestyle{fancy}
\fancyhf{} 
\fancyhead[L]{\includegraphics[height=1.5cm]{portada_img/izq.png}}
\fancyhead[R]{\includegraphics[height=1.5cm]{portada_img/der.png}}
\fancyfoot[C]{\thepage} 
\renewcommand{\headrulewidth}{0.4pt} 
\renewcommand{\footrulewidth}{0pt}  

% --- BIBLIOGRAFÍA ---
\usepackage[backend=biber, style=apa, sortcites, url=true]{biblatex}
\addbibresource{referencias.bib}

% --- HIPERVÍNCULOS ---
\usepackage{hyperref}
\hypersetup{
    colorlinks=true,
    linkcolor=blue,
    citecolor=blue,
    filecolor=magenta,
    urlcolor=cyan
}

\begin{document}

    \thispagestyle{empty} % Sin encabezado ni pie en la portada

% --- ENCABEZADO DE PORTADA (TABLA DE 3 COLUMNAS) ---
\begin{center}
    % La tabla usa 'm' para centrar verticalmente las imágenes y el texto
    \begin{tabular}{m{0.15\textwidth} m{0.60\textwidth} m{0.15\textwidth}}
        % Logo Izquierdo (UNAM)
        \includegraphics[width=2cm]{portada_img/izq.png} & 
        
        % Título Central
        \centering\Huge\textbf{Universidad Nacional Autónoma de México} &
        
        % Logo Derecho (Facultad)
        \raggedleft\includegraphics[width=2cm]{portada_img/der.png}
    \end{tabular}
\end{center}

\vspace{0.3cm}

% --- INFORMACIÓN DEL REPORTE ---
\begin{center}
    {\LARGE \textbf{Facultad de Ingeniería}}\\[1cm]

    {\LARGE \textbf{Integrantes:}}\\[0.3cm]
    {\Large Espinoza Matamoros Percival Ulises - 320025561} \\[0.4cm]
    {\Large Flores Colin Victor Jaziel - 320266083} \\[0.4cm]
    {\Large Lara Hernandez Angel Husiel - 320060829} \\[1cm]

    {\LARGE \textbf{Laboratorio de Microcomputadoras}} \\[1cm]

	{\LARGE Grupo: 06 - Semestre: 2026-2} \\[1cm]
    
    {\LARGE \textbf{Practica 1:}}\\[0.5cm]

    {\LARGE {Introducción de las arquitecturas ARM empleando Raspberry Pi}}\\[1cm]

    {\LARGE \textbf{Profesor:}}\\[0.2cm]
    {\Large Ing. Moises Melendez Reyes}\\[1cm]

        

    
    
    {\Large \textbf{Fecha de Entrega:}}\\[0.2cm]
    {\large 1 de Marzo del 2026}
\end{center}

    \newpage
    \setcounter{page}{1}

    \section{Objetivo:}
    Aprender la estructura de los procesadores con arquitectura ARM, utilizar la plataforma Raspberry Pi, los entornos para programar; desarrollar algoritmos con las instrucciones en lenguaje ensamblador, controlar directamente los recursos del microprocesador; editar, compilar, ensamblar, simular y ejecutar programas en Raspberry Pi.

    \section*{Actividad 1}
    Seguir el procedimiento indicado en el apartado cuarto de manual de tutoriales, escribir, comentar y ensamblar y ejecutar el siguiente programa; explicar qué hace.

    \subsection*{Propuesta de solución}
    Se propone cargar cada operando directamente en un registro mediante
    \textbf{direccionamiento inmediato} (\texttt{MOV Rn, \#valor}), de modo que el dato
    viaja desde la instrucción misma hacia el banco de registros sin pasar por memoria.
    Una vez que ambos operandos residen en R1 y R2, la ALU calcula la suma con
    \texttt{ADD R3, R2, R1} y deposita el resultado en R3; finalmente, dicho valor se
    copia a R0 (registro de retorno por convención ABI) antes de invocar la llamada al
    sistema de salida. El diagrama de flujo que representa este proceso es:
    
    \begin{center}
    \begin{tikzpicture}[node distance=1.5cm]
        \node (start) [startstop] {Inicio};
        \node (pro1) [process, below of=start] {Cargar $0x19$ en R1};
        \node (pro2) [process, below of=pro1] {Cargar $53$ en R2};
        \node (pro3) [process, below of=pro2] {R3 = R2 + R1};
        \node (pro4) [process, below of=pro3] {Mover R3 a R0};
        \node (stop) [startstop, below of=pro4] {Fin (Syscall Exit)};

        \draw [arrow] (start) -- (pro1);
        \draw [arrow] (pro1) -- (pro2);
        \draw [arrow] (pro2) -- (pro3);
        \draw [arrow] (pro3) -- (pro4);
        \draw [arrow] (pro4) -- (stop);
    \end{tikzpicture}
    \end{center}

    \newpage

    \textbf{Desarrollo}
    Se transcribió y compiló el código fuente. Se utilizó GDB para la depuración y verificación de los registros de la CPU.
    
    \begin{lstlisting}[language={[ARM]Assembler}, caption=Código de la Actividad 1]
    .global _start      @ Hace visible la etiqueta _start para el enlazador (linker)

    _start:             @ Punto de entrada del programa
        MOV R1, #0x19   @ Carga el valor hexadecimal 19 (25 decimal) en el registro R1
        MOV R2, #53     @ Carga el valor decimal 53 en el registro R2
        
        ADD R3, R2, R1  @ Suma: R3 = R2 + R1. Resultado esperado: 78
        
        MOV R0, R3      @ Mueve el resultado (78) al registro R0 (registro de retorno)
        MOV R7, #1      @ Carga el valor 1 en R7. En Linux, 1 significa "Syscall Exit"
        SVC 0           @ Supervisor Call: Llama al Kernel para ejecutar la salida
    \end{lstlisting}

    \begin{figure}[H]
        \centering
        \includegraphics[width=0.7\textwidth]{A1_2.png}
        \caption{Proceso de ensamblado, enlazado e inicio de GDB en Raspberry Pi.}
    \end{figure}

    \begin{figure}[H]
        \centering
        \includegraphics[width=0.7\textwidth]{A1_1.png}
        \caption{Inspección de registros en GDB mostrando R0 y R3 con el valor de 78 (0x4E).}
    \end{figure}

    \textbf{Análisis de resultados}

    El programa utiliza el \textbf{modo de direccionamiento inmediato} para cargar constantes estáticas en los registros internos del procesador (\texttt{MOV R1, \#0x19} y \texttt{MOV R2, \#53}). El flujo interno de los datos envía el contenido de los registros R1 y R2 a la Unidad Aritmético Lógica (ALU) a través de la instrucción \texttt{ADD}, la cual deposita el resultado de la suma matemática en el registro R3.
    Se comprobó que el flujo y la lógica son correctos, ya que la suma de $25 + 53$ arroja $78$ en decimal, equivalente a \texttt{0x4E} en hexadecimal, valor que se observa claramente almacenado en los registros durante la ejecución controlada con GDB.

    \newpage

    \section*{Actividad 2}
    Seguir el procedimiento indicado en el apartado cuarto de manual de tutoriales, escribir, comentar, ensamblar y ejecutar el siguiente programa; explicar qué hace.

    \subsection*{Propuesta de solución}
    El problema requiere comparar dos valores y seleccionar un mensaje de salida según
    el resultado. Se propone usar la instrucción \texttt{SUBS} para restar ambos valores
    y actualizar simultáneamente la bandera \textbf{Zero (Z)} del registro \texttt{CPSR}:
    si el resultado es cero, los datos son iguales (Z\,=\,1); si no, son diferentes (Z\,=\,0).
    A partir de ese estado de bandera, las instrucciones \texttt{BEQ} y \texttt{BNE}
    dirigen el flujo hacia la rama correspondiente de entrada/salida.
    El dato transita así del banco de registros hacia la ALU (para la resta), luego al
    \texttt{CPSR} (para fijar Z) y finalmente al periférico de salida estándar mediante
    la llamada al sistema. El diagrama de flujo resultante es:

    \begin{center}
    \begin{tikzpicture}[node distance=1.5cm]
        \node (start) [startstop] {Inicio};
        \node (pro1) [process, below of=start] {R0 = 5,\ R1 = 1};
        \node (pro2) [process, below of=pro1] {SUBS: R3 = R0 $-$ R1};
        \node (dec1) [decision, below of=pro2, yshift=-0.5cm] {¿R3 == 0?};

        \node (pro3) [io, below of=dec1, xshift=-3.2cm, yshift=-1.5cm,
                      minimum width=2.8cm, minimum height=0.8cm, align=center]
                     {Imprimir:\\''Datos diferentes''};
        \node (pro4) [io, below of=dec1, xshift=3.2cm, yshift=-1.5cm,
                      minimum width=2.8cm, minimum height=0.8cm, align=center]
                     {Imprimir:\\''Datos iguales''};
        \node (stop) [startstop, below of=dec1, yshift=-4.5cm] {Fin};

        \draw [arrow] (start) -- (pro1);
        \draw [arrow] (pro1) -- (pro2);
        \draw [arrow] (pro2) -- (dec1);
        \draw [arrow] (dec1.west) -- node[above]{\small No (Z=0)} (dec1.west -| pro3.north) -- (pro3.north);
        \draw [arrow] (dec1.east) -- node[above]{\small Sí (Z=1)} (dec1.east -| pro4.north) -- (pro4.north);
        \draw [arrow] (pro3.south) |- (stop);
        \draw [arrow] (pro4.south) |- (stop);
    \end{tikzpicture}
    \end{center}

    \newpage

    \textbf{Desarrollo}
    
    \begin{lstlisting}[language={[ARM]Assembler}, caption=Código de la Actividad 2]
    .text                   
    .global _start          

    _start:
        MOV R0, #5          @ Carga el valor 5 en R0
        MOV R1, #0x01       @ Carga el valor 1 en R1
        SUBS R3, R0, R1     @ Resta R1 a R0 (5-1), guarda en R3 y actualiza banderas
        
        BEQ igual           @ Si Z=1, salta a etiqueta 'igual'
        BNE diferente       @ Si Z=0, salta a 'diferente'

    igual:                  
        MOV R0, #1          @ Descriptor de archivo 1 (Salida estandar / pantalla)
        LDR R1, =texto1     @ Carga la direccion de 'texto1'
        MOV R2, #14         @ Longitud del mensaje
        MOV R7, #4          @ Syscall 4 (Write)
        SVC 0               
        B fin               @ Salto al final

    diferente:              
        MOV R0, #1          @ Descriptor de archivo 1
        LDR R1, =texto2     @ Carga la direccion de 'texto2'
        MOV R2, #17         @ Longitud del mensaje
        MOV R7, #4          @ Syscall 4 (Write)
        SVC 0               

    fin:                    
        MOV R0, R3          
        MOV R7, #1          @ Syscall 1 (Exit)
        SVC 0               

    .data                   
        texto1: .asciz "Datos iguales\n"      
        texto2: .asciz "Datos diferentes\n"   
    \end{lstlisting}

    \begin{figure}[H]
        \centering
        \includegraphics[width=0.7\textwidth]{A2_1.png}
        \caption{Inspección de registros antes del salto condicional.}
    \end{figure}

    \begin{figure}[H]
        \centering
        \includegraphics[width=0.7\textwidth]{A2_2.png}
        \caption{Ejecución del programa evaluando la condición de desigualdad.}
    \end{figure}

    \newpage

    \textbf{Análisis de resultados}
    
    Se implementó un algoritmo de control de flujo usando saltos condicionales (\texttt{BEQ} y \texttt{BNE}). La instrucción clave aquí es \texttt{SUBS}, la cual no solo realiza la resta mediante la ALU, sino que interactúa con el registro \texttt{CPSR} (Current Program Status Register) para actualizar la bandera Zero (Z). Como $5 - 1 = 4$, la bandera Z se estableció en 0, activando la rama del salto \texttt{BNE}. Adicionalmente, el programa utiliza el periférico de salida estándar (pantalla) a través de la llamada al sistema de Linux (\texttt{SVC 0} con R7=4), mandando la cadena de texto almacenada en la memoria \texttt{.data} hacia la terminal.

    \section*{Actividad 3}
    Empleando el IDE Code::Block, seleccionar 10 instrucciones, formalizar un programa; comprobar el funcionamiento (agregar las directivas correspondientes).

    \begin{enumerate}[label=\alph*.]
        \item Reportar el resultado esperado y el obtenido.
    \end{enumerate}


    \newpage

    \section*{Actividad 4}
    Tomando como base el programa de la actividad 1, para que obtenga el promedio de dos números de 8 bits; utilizar Code::Blocks para todo el proceso.

    \subsection*{Propuesta de solución}
    Para obtener el promedio de dos números de 8 bits se parte de la identidad
    $\overline{x} = (A + B) \div 2$. La suma se realiza con \texttt{ADD} igual que
    en la Actividad~1, y la división entre 2 se sustituye por un desplazamiento
    lógico a la derecha de un bit (\texttt{LSR \#1}), operación equivalente y más
    eficiente en arquitectura ARM. El diagrama de flujo resultante es:

    \begin{center}
    \begin{tikzpicture}[node distance=1.5cm]
        \node (start) [startstop] {Inicio};
        \node (pro1) [process, below of=start] {Cargar Dato 1 (80) en R1};
        \node (pro2) [process, below of=pro1] {Cargar Dato 2 (20) en R2};
        \node (pro3) [process, below of=pro2] {Suma: R3 = R1 + R2};
        \node (pro4) [process, below of=pro3] {Desplazamiento a la derecha: R0 = R3 $\gg$ 1};
        \node (stop) [startstop, below of=pro4] {Fin};

        \draw [arrow] (start) -- (pro1);
        \draw [arrow] (pro1) -- (pro2);
        \draw [arrow] (pro2) -- (pro3);
        \draw [arrow] (pro3) -- (pro4);
        \draw [arrow] (pro4) -- (stop);
    \end{tikzpicture}
    \end{center}

    \textbf{Desarrollo}

    \begin{lstlisting}[language={[ARM]Assembler}, caption=Código de la Actividad 4]
    .text
    .global main

    main:
        MOV R1, #80         @ Dato 1 
        MOV R2, #20         @ Dato 2 
        
        ADD R3, R1, R2      @ R3 = 80 + 20 = 100
        
        @ Para dividir entre 2 usamos LSR (Logical Shift Right)
        LSR R0, R3, #1      @ Desplaza bits a la derecha 1 vez. 
        
        @ El resultado (50) ya esta en R0 listo para devolverlo
        MOV R7, #1          @ Salida
        SVC 0
    \end{lstlisting}

    \begin{figure}[H]
        \centering
        \includegraphics[width=0.8\textwidth]{A4.png}
        \caption{Verificación de registros en Code::Blocks para la Actividad 4, mostrando R0=0x32 (50).}
    \end{figure}

    \textbf{Análisis de resultados}

    El objetivo se cumplió promediando dos números enteros de manera eficiente. En lugar de utilizar una instrucción de división pesada, se manipuló la arquitectura mediante un \textbf{Logical Shift Right (LSR)}. En la arquitectura ARM, el flujo de datos pasa a través de un desplazador de barril (\textit{barrel shifter}) antes de la ALU. Mover los bits una posición a la derecha equivale matemáticamente a dividir entre 2 ($100 \gg 1 = 50$). El depurador en Code::Blocks muestra en la imagen el registro R0 con el valor correcto de \texttt{0x32} (50 en decimal).


    \section*{Actividad 5}
    Emplear el IDE Code::Block, escribir, comentar, compilar y ejecutar el siguiente programa.

    \subsection*{Propuesta de solución}
    El problema consiste en implementar un contador cíclico que incremente su valor de
    0 a 9 y luego lo decremente de 9 a 0, repitiéndose indefinidamente.
    Se propone almacenar el contador en R0, el límite superior (9) en R1 y el límite
    inferior (0) en R2. La solución emplea dos bucles etiquetados (\texttt{loop1} y
    \texttt{loop2}) controlados por la instrucción \texttt{CMP}, la cual resta
    internamente los operandos sin guardar resultado, actualizando solo las banderas del
    \texttt{CPSR}. El flujo de datos es el siguiente: R0 avanza por \texttt{ADD} hasta
    alcanzar R1, momento en que \texttt{BNE} deja de redirigir el \texttt{PC} a
    \texttt{loop1} y el control pasa a \texttt{loop2}; desde ahí, \texttt{ADD R0,\#-1}
    decrementa R0 hasta igualar R2, y \texttt{BEQ} devuelve el \texttt{PC} a
    \texttt{loop1}. El diagrama de flujo que describe este comportamiento cíclico es:

    \begin{center}
    \begin{tikzpicture}[node distance=1.8cm]
        \node (start) [startstop] {Inicio};
        \node (pro1) [process, below of=start] {R0=0, R1=9 (Sup), R2=0 (Inf)};
        \node (loop1) [process, below of=pro1] {loop1: R0 = R0 + 1};
        \node (dec1) [decision, below of=loop1, yshift=-1.0cm] {¿R0 == R1?};
        
        \node (loop2) [process, right of=dec1, xshift=3cm] {loop2: R0 = R0 - 1};
        \node (dec2) [decision, above of=loop2, yshift=1cm] {¿R0 == R2?};

        \draw [arrow] (start) -- (pro1);
        \draw [arrow] (pro1) -- (loop1);
        \draw [arrow] (loop1) -- (dec1);
        
        % Loop ascendente
        \draw [arrow] (dec1.west) -- ++(-1.5,0) |- node[anchor=east, pos=0.25] {No} (loop1.west);
        
        % Transición a descendente
        \draw [arrow] (dec1.east) -- node[anchor=south] {Sí} (loop2.west);
        \draw [arrow] (loop2) -- (dec2);
        
        % Loop descendente
        \draw [arrow] (dec2.east) -- ++(1.5,0) |- node[anchor=west, pos=0.25] {No} (loop2.east);
        
        % Regreso a ascendente
        \draw [arrow] (dec2.west) -- node[anchor=south] {Sí} (loop1.east);
    \end{tikzpicture}
    \end{center}

    \textbf{Desarrollo}

    \begin{lstlisting}[language={[ARM]Assembler}, caption=Código de la Actividad 5]
        /* Actividad 5: Bucle Ascendente y Descendente */

        .text
        .global main

        main:
        MOV R0, #0          @ R0 sera nuestro contador, inicia en 0
        MOV R1, #9          @ R1 es el limite superior (9)
        MOV R2, #0          @ R2 es el limite inferior (0)

        loop1:                  @ Etiqueta para subir
        ADD R0, R0, #1      @ Incrementa R0 en 1
        CMP R1, R0          @ Compara si llegamos al limite superior (9)
        BNE loop1           @ Si NO es igual, repite loop1

        loop2:                  @ Etiqueta para bajar
        ADD R0, R0, #-1     @ Decrementa R0 en 1
        CMP R2, R0          @ Compara si llegamos al limite inferior (0)
        BEQ loop1           @ Si es igual a 0, salta de nuevo a loop1
        B loop2             @ Si no es 0, sigue bajando
    \end{lstlisting}

    \begin{figure}[H]
        \centering
        \includegraphics[width=0.8\textwidth]{A5.png}
        \caption{Monitoreo de bucle en Code::Blocks, mostrando a R0 en la iteración de bajada.}
    \end{figure}

    \textbf{Análisis de resultados}

    Este programa demuestra el funcionamiento interno del \textbf{Contador de Programa (PC)} ante instrucciones de bifurcación. El ciclo infinito es gobernado por la instrucción de comparación virtual \texttt{CMP}, que resta internamente los registros sin almacenar el resultado, solo para manipular el \texttt{CPSR}. Al combinarse con ramas relativas (\texttt{BNE}, \texttt{BEQ}, \texttt{B}), la CPU salta las direcciones de memoria hacia atrás, repitiendo el proceso lógico. En la evidencia visual se observa que el contador principal \texttt{R0} ha operado correctamente dentro de los márgenes limitantes de los registros \texttt{R1} y \texttt{R2}.

    \section*{Actividad 6}

    Realizar un programa que inicie activando el bit menos significativo de un registro y recorra de posición hacia el bit más significativo (solo un bit estará activado); usar el IDE Code::Blocks.


    \newpage

    \section*{Actividad 7}

    Escribir un programa que realice la suma de dos números de 32 bits y almacene el resultado en memoria empleando las direcciones que considere el resultado del acarreo en caso de existir.


    \newpage
    \section*{Actividad 8}

    Escribir un programa que realice la suma de dos números de 64 bits y almacene el resultado en memoria empleando las direcciones que considere el resultado del acarreo en caso de existir.


    \newpage
    \section*{Actividad 9}

    Realizar un programa que obtenga el factorial de un número de 8 bits.


    \newpage
    \section*{Actividad 10}

    Implementar con instrucciones en lenguaje ensamblador la sentencia:


    \newpage
    \section{Conclusiones:}



    \begin{itemize}

        \item \textbf{Espinoza Matamoros Percival Ulises:}

        \item \textbf{Flores Colin Victor Jaziel:}

        \item \textbf{Lara Hernandez Angel Husiel:}

    \end{itemize}

    \newpage

    \nocite{*}

    \printbibliography[heading=bibintoc, title={Referencias}]


\end{document}