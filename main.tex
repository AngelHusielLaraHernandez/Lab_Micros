\documentclass[letterpaper,12pt]{article}

% --- PAQUETES DE IDIOMA
\usepackage[spanish,mexico]{babel}
\usepackage[utf8]{inputenc}
\usepackage{csquotes}

% --- PAQUETES MATEMÁTICOS --
\usepackage{amsmath}
\usepackage{amssymb}
\usepackage{cancel}
\usepackage{mathrsfs}

% --- PAQUETES GRÁFICOS Y DE DISEÑO ---
\usepackage{graphicx}
\usepackage{float}
\usepackage{subcaption}
\usepackage{xcolor}
\usepackage{makecell}
\usepackage{array}
\usepackage{tikz}
\usetikzlibrary{shapes.geometric, positioning}
\usetikzlibrary{shapes.multipart, positioning, arrows.meta, calc, shadows}


% --- LISTAS ---
\usepackage{enumitem}

% --- PAQUETES DE CÓDIGO ---
\usepackage{listings}
% Configuración del estilo de listings
% Paleta de colores estilo VSCode
\definecolor{codebg}{rgb}{0.95,0.95,0.95}   % gris claro de fondo
\definecolor{keyword}{rgb}{0.0,0.2,0.6}     % azul oscuro
\definecolor{comment}{rgb}{0.25,0.5,0.35}   % verde
\definecolor{string}{rgb}{0.6,0.0,0.0}      % rojo oscuro
\definecolor{number}{rgb}{0.5,0.0,0.5}      % morado

% Paleta de colores personalizada
\definecolor{codegray}{rgb}{0.5,0.5,0.5}
\definecolor{codepurple}{rgb}{0.58,0,0.82}
\definecolor{backcolour}{rgb}{0.95,0.95,0.95}

% Configuración del estilo
\lstset{
  backgroundcolor=\color{backcolour},   % Fondo gris claro
  basicstyle=\ttfamily\small,           % Fuente monoespaciada
  keywordstyle=\color{blue}\bfseries,   % Palabras clave en azul y negrita
  stringstyle=\color{codepurple},       % Cadenas en morado
  commentstyle=\color{codegray}\itshape,% Comentarios grises en cursiva
  numbers=left,                         % Números de línea a la izquierda
  numberstyle=\tiny\color{codegray},    % Estilo de los números
  frame=single,                         % Marco alrededor
  rulecolor=\color{black},              % Color del marco
  breaklines=true,                      % Ajuste automático de línea
  tabsize=4,                            % Tabulación de 4 espacios
  showstringspaces=false                % No mostrar espacios raros
}


% --- RUTAS DE IMÁGENES ---
% LaTeX buscará imágenes en estas carpetas automáticamente
\graphicspath{{img/}{portada_img/}} 

% --- DIBUJO DE CIRCUITOS ---
\usepackage{pgf,tikz}
\usetikzlibrary{decorations.pathreplacing, calc, positioning}

% Definimos el color azul similar al de la imagen
\definecolor{armblue}{RGB}{0, 113, 188}

% --- CONFIGURACIÓN DE PÁGINA (GEOMETRÍA) ---
\usepackage[
    left=25mm, 
    right=25mm,
    top=35mm,
    bottom=30mm,
    headheight=77pt, 
    papersize={21.59cm,27.94cm}
]{geometry}

% --- FORMATO DE TEXTO ---
\renewcommand{\baselinestretch}{1.25} % Interlineado 1.25
\usepackage{parskip} % Espacio entre párrafos sin sangría
\usepackage{lipsum}  % Texto de relleno

% --- VARIABLES GLOBALES ---
\newcommand{\materia}{}

% --- ENCABEZADOS Y PIES DE PÁGINA ---
\usepackage{fancyhdr}
\pagestyle{fancy}
\fancyhf{} % Limpia encabezados y pies por defecto

% CONFIGURACIÓN DE LOS ESCUDOS EN EL ENCABEZADO
\fancyhead[L]{\includegraphics[height=1.5cm]{portada_img/izq.png}} 
\fancyhead[R]{\includegraphics[height=1.5cm]{portada_img/der.png}} 

\fancyfoot[C]{\thepage} % Número de página al centro abajo
\renewcommand{\headrulewidth}{0.4pt} % Línea horizontal bajo el encabezado
\renewcommand{\footrulewidth}{0pt}   % Sin línea en el pie

% --- BIBLIOGRAFÍA ---
\usepackage[backend=biber, style=apa, sortcites, url=true]{biblatex}
\addbibresource{referencias.bib}

% --- HIPERVÍNCULOS (Siempre al final del preámbulo) ---
\usepackage{hyperref}
\hypersetup{
    colorlinks=true,
    linkcolor=blue,
    citecolor=blue,
    filecolor=magenta,
    urlcolor=cyan
}

% ---------------------------------------------------------
% INICIO DEL DOCUMENTO
% ---------------------------------------------------------

\begin{document}

    % 1. INSERTAR PORTADA (portada.tex)
    \thispagestyle{empty} % Sin encabezado ni pie en la portada

% --- ENCABEZADO DE PORTADA (TABLA DE 3 COLUMNAS) ---
\begin{center}
    % La tabla usa 'm' para centrar verticalmente las imágenes y el texto
    \begin{tabular}{m{0.15\textwidth} m{0.60\textwidth} m{0.15\textwidth}}
        % Logo Izquierdo (UNAM)
        \includegraphics[width=2cm]{portada_img/izq.png} & 
        
        % Título Central
        \centering\Huge\textbf{Universidad Nacional Autónoma de México} &
        
        % Logo Derecho (Facultad)
        \raggedleft\includegraphics[width=2cm]{portada_img/der.png}
    \end{tabular}
\end{center}

\vspace{0.3cm}

% --- INFORMACIÓN DEL REPORTE ---
\begin{center}
    {\LARGE \textbf{Facultad de Ingeniería}}\\[1cm]

    {\LARGE \textbf{Integrantes:}}\\[0.3cm]
    {\Large Espinoza Matamoros Percival Ulises - 320025561} \\[0.4cm]
    {\Large Flores Colin Victor Jaziel - 320266083} \\[0.4cm]
    {\Large Lara Hernandez Angel Husiel - 320060829} \\[1cm]

    {\LARGE \textbf{Laboratorio de Microcomputadoras}} \\[1cm]

	{\LARGE Grupo: 06 - Semestre: 2026-2} \\[1cm]
    
    {\LARGE \textbf{Practica 1:}}\\[0.5cm]

    {\LARGE {Introducción de las arquitecturas ARM empleando Raspberry Pi}}\\[1cm]

    {\LARGE \textbf{Profesor:}}\\[0.2cm]
    {\Large Ing. Moises Melendez Reyes}\\[1cm]

        

    
    
    {\Large \textbf{Fecha de Entrega:}}\\[0.2cm]
    {\large 1 de Marzo del 2026}
\end{center}

    \newpage
    \setcounter{page}{1} % Reiniciar contador en 1

    % 2. CONTENIDO
    \section{Objetivo:}
    Aprender la estructura de los procesadores con arquitectura ARM, utilizar la plataforma Raspberry Pi, los entornos para programar; desarrollar algoritmos con las instrucciones en lenguaje ensamblador, controlar directamente los recursos del microprocesador; editar, compilar, ensamblar, simular y ejecutar programas en Raspberry Pi.

    \section*{Actividad 1}
    Seguir el procedimiento indicado en el apartado cuarto de manual de tutoriales, escribir,
    comentar y ensamblar y ejecutar el siguiente programa; explicar que hace:

    \section*{Actividad 2}
    Seguir el procedimiento indicado en el apartado cuarto de manual de tutoriales, escribir,
    comentar, ensamblar y ejecutar el siguiente programa; explicar que hace.

    \section*{Actividad 3}
    Empleando el IDE Code::Block, seleccionar 10 instrucciones, formalizar un programa;
    comprobar el funcionamiento (agregar las directivas correspondientes).

    \begin{enumerate}[label=\alph*.]
        \item Reportar el resultado esperado y el obtenido.
    \end{enumerate}

    \section*{Actividad 4}
    Tomando como base el programa de la actividad 1, para que obtenga el promedio de dos
    números de 8 bits; utilizar Code::Blocks para todo el proceso.

    \section*{Actividad 5}
    Emplear el IDE Code::Block, escribir, comentar, compilar y ejecutar el siguiente programa.

    \section*{Actividad 6}
    Realizar un programa que inicie activando el bit menos significativo de un registro y
    recorra de posición hacia el bit más significativo (solo un bit estará activado); usar el
    IDE Code::Blocks.

    \section*{Actividad 7}
    Escribir un programa que realice la suma de dos números de 32 bits y almacene el resultado
    en memoria empleando las direcciones que considere el resultado del acarreo en caso de existir.

    \section*{Actividad 8}
    Escribir un programa que realice la suma de dos números de 64 bits y almacene el resultado
    en memoria empleando las direcciones que considere el resultado del acarreo en caso de existir.

    \section*{Actividad 9}
    Realizar un programa que obtenga el factorial de un número de 8 bits.

    \section*{Actividad 10}
    Implementar con instrucciones en lenguaje ensamblador la sentencia:

    \section{Conclusiones:}

    \begin{itemize}
        \item \textbf{Espinoza Matamoros Percival Ulises:}
        \item \textbf{Flores Colin Victor Jaziel:}
        \item \textbf{Lara Hernandez Angel Husiel:}
    \end{itemize}
    \newpage
    \nocite{*} 
    \printbibliography[heading=bibintoc, title={Referencias}]

\end{document}